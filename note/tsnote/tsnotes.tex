\documentclass{tsnotes}
\usepackage{ctex}
\usepackage{float}
\usepackage{lipsum}

\begin{document}

\tstitle{\LaTeX~Note}

\section{First Section——第一节}
\lipsum[1]


\section{Second Section——第二节}
中文宏包测试               \par 
{\songti   这是宋体的样式。} 
{\heiti    这是黑体的样式。} 
{\fangsong 这是仿宋的样式。} 
{\kaishu   这是楷书的样式。} \par

\section{Third Section——第三节}
\lipsum[1-2]
\section{Fourth Section——第四节}
\begin{lstlisting}[title=代码块1,language=Python]
# Definition for singly-linked list.
# class ListNode:
#     def __init__(self, x):
#         self.val = x
#         self.next = None

class Solution:
    def mergeTwoLists(self, l1, l2):
        """
        :type l1: ListNode
        :type l2: ListNode
        :rtype: ListNode
        """
        head = ListNode(0)
        l3 = head
        
        while l1 and l2:
            if l1.val > l2.val:
                l3.next = ListNode(l2.val)
                l3 = l3.next
                l2 = l2.next
            else:
                l3.next = ListNode(l1.val)
                l3 = l3.next
                l1 = l1.next
        if l1:
            while l1:
                l3.next = ListNode(l1.val)
                l3 = l3.next
                l1 = l1.next
        if l2:
            while l2:
                l3.next = ListNode(l2.val)
                l3 = l3.next
                l2 = l2.next
        return head.next
\end{lstlisting}


\subsection*{Example}

If you ever wanted to know the area of a circle, you could compute


this integral:

\begin{equation}
  A = \int_0^R \int_0^{2\pi} r \,\mathrm{d}\varphi\,\mathrm{d}r
    = \pi R^2.
\end{equation}

\fixme{It's great to make a note of things that need to be fixed.}

\end{document}